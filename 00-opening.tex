\documentclass[sigconf]{acmart}

\usepackage{booktabs} % For formal tables


% Copyright
%\setcopyright{none}
%\setcopyright{acmcopyright}
%\setcopyright{acmlicensed}
\setcopyright{rightsretained}
%\setcopyright{usgov}
%\setcopyright{usgovmixed}
%\setcopyright{cagov}
%\setcopyright{cagovmixed}


% DOI
\acmDOI{10.475/123_4}

% ISBN
\acmISBN{123-4567-24-567/08/06}

%Conference
\acmConference[WOODSTOCK'97]{ACM Woodstock conference}{July 1997}{El
  Paso, Texas USA}
\acmYear{1997}
\copyrightyear{2016}


\acmArticle{4}
\acmPrice{15.00}

% These commands are optional
%\acmBooktitle{Transactions of the ACM Woodstock conference}
\editor{Jennifer B. Sartor}
\editor{Theo D'Hondt}
\editor{Wolfgang De Meuter}


\begin{document}
\title{SIG Proceedings Paper in LaTeX Format}
\titlenote{Produces the permission block, and
  copyright information}
\subtitle{Extended Abstract}
\subtitlenote{The full version of the author's guide is available as
  \texttt{acmart.pdf} document}


\author{Denise M. Case}
%\authornote{Dr.~Trovato insisted his name be first.}
\orcid{0000-0001-6165-7389}
\affiliation{%
  \institution{Northwest Missouri State University}
  \streetaddress{800 University Drive}
  \city{Maryville}
  \state{Missouri}
  \postcode{64468}
}
\email{dcase@nwmissouri.edu}

\author{G.K.M. Tobin}
\authornote{The secretary disavows any knowledge of this author's actions.}
\affiliation{%
  \institution{Institute for Clarity in Documentation}
  \streetaddress{P.O. Box 1212}
  \city{Dublin}
  \state{Ohio}
  \postcode{43017-6221}
}
\email{webmaster@marysville-ohio.com}

\author{Lars Th{\o}rv{\"a}ld}
\authornote{This author is the
  one who did all the really hard work.}
\affiliation{%
  \institution{The Th{\o}rv{\"a}ld Group}
  \streetaddress{1 Th{\o}rv{\"a}ld Circle}
  \city{Hekla}
  \country{Iceland}}
\email{larst@affiliation.org}

\author{Valerie B\'eranger}
\affiliation{%
  \institution{Inria Paris-Rocquencourt}
  \city{Rocquencourt}
  \country{France}
}
\author{Aparna Patel}
\affiliation{%
 \institution{Rajiv Gandhi University}
 \streetaddress{Rono-Hills}
 \city{Doimukh}
 \state{Arunachal Pradesh}
 \country{India}}
\author{Huifen Chan}
\affiliation{%
  \institution{Tsinghua University}
  \streetaddress{30 Shuangqing Rd}
  \city{Haidian Qu}
  \state{Beijing Shi}
  \country{China}
}

\author{Charles Palmer}
\affiliation{%
  \institution{Palmer Research Laboratories}
  \streetaddress{8600 Datapoint Drive}
  \city{San Antonio}
  \state{Texas}
  \postcode{78229}}
\email{cpalmer@prl.com}

\author{John Smith}
\affiliation{\institution{The Th{\o}rv{\"a}ld Group}}
\email{jsmith@affiliation.org}

\author{Julius P.~Kumquat}
\affiliation{\institution{The Kumquat Consortium}}
\email{jpkumquat@consortium.net}

% The default list of authors is too long for headers.
\renewcommand{\shortauthors}{B. Trovato et al.}


\begin{abstract}
This paper provides a sample of a \LaTeX\ document which conforms,
somewhat loosely, to the formatting guidelines for
ACM SIG Proceedings.\footnote{This is an abstract footnote}
\end{abstract}

%
% The code below should be generated by the tool at
% http://dl.acm.org/ccs.cfm
% Please copy and paste the code instead of the example below.
%
\begin{CCSXML}
<ccs2012>
 <concept>
  <concept_id>10010520.10010553.10010562</concept_id>
  <concept_desc>Computer systems organization~Embedded systems</concept_desc>
  <concept_significance>500</concept_significance>
 </concept>
 <concept>
  <concept_id>10010520.10010575.10010755</concept_id>
  <concept_desc>Computer systems organization~Redundancy</concept_desc>
  <concept_significance>300</concept_significance>
 </concept>
 <concept>
  <concept_id>10010520.10010553.10010554</concept_id>
  <concept_desc>Computer systems organization~Robotics</concept_desc>
  <concept_significance>100</concept_significance>
 </concept>
 <concept>
  <concept_id>10003033.10003083.10003095</concept_id>
  <concept_desc>Networks~Network reliability</concept_desc>
  <concept_significance>100</concept_significance>
 </concept>
</ccs2012>
\end{CCSXML}

\ccsdesc[500]{Computer systems organization~Embedded systems}
\ccsdesc[300]{Computer systems organization~Redundancy}
\ccsdesc{Computer systems organization~Robotics}
\ccsdesc[100]{Networks~Network reliability}


\keywords{ACM proceedings, \LaTeX, text tagging}


\maketitle

% comment following section for production
\section{Opening}

50th ACM Technical Symposium on Computer Science Education! 

\begin{itemize}
	\item \url{https://sigcse2019.sigcse.org/}
  \item \url{https://sigcse2019.sigcse.org/SIGCSE_2019_CFP.pdf}
\end{itemize}



February 27 - March 2, 2019, Minneapolis, MN, USA

PAPERS (6 pp. max; 25 min. presentation) Papers describe an educational research project, classroom experience, teaching
technique, curricular initiative, or pedagogical tool. All papers should explicitly state their motivating questions, relate to relevant
literature, and contain an analysis of the effectiveness of the interventions, including limitations. Initial submissions must be anonymous.
Note that an ABSTRACT SUBMISSION is now required for all papers and it is due a week before the full paper is due. 

\begin{itemize}
	\item CS Education Research papers should adhere to rigorous standards, describing hypotheses, methods, results, and
limitations as is typical for research studies. These normally focus on topics relevant to computing education with emphasis on
educational goals and knowledge units/topics relevant to computing education with statistical rigor; methods or techniques in
computing education; evaluation of pedagogical approaches; and studies of the many populations engaged in computing
education, including (but not limited to) students, instructors, and issues of gender, diversity, and underrepresentation.

	\item Experience Reports and Tools papers should carefully describe a computer science education intervention and its context,
and provide a rich reflection on what worked, what didn’t, and why. This track accepts experience reports, teaching
techniques, and pedagogical tools. All papers in this track should provide enough detail for adoption by others.

	\item New Curricula, Programs, Degrees and Position Papers. Papers about curricula, programs and degrees should describe
the motivating context before the new initiative was undertaken, what it took to put the initiative into place, what the impact has
been, and suggestions for others wishing to adopt it. Position papers are meant to engender fruitful academic discussion by
presenting a defensible opinion about a CS education topic, substantiated with evidence.
\end{itemize}

\input{10-introduction.tex}
\input{20-related-work.tex}
\input{30-section.tex}
\input{40-section.tex}
\input{50-section.tex}
\input{80-results.tex}
\input{90_conclusions.tex}

\bibliographystyle{ACM-Reference-Format}
\bibliography{sigcse2018}

\end{document}